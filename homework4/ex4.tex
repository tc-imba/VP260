\documentclass{article}
\usepackage{enumerate}
\usepackage{amsmath}
\usepackage{amssymb}
\usepackage{graphicx}
\usepackage{subfigure}
\usepackage{geometry}
\usepackage{caption}
\usepackage{stmaryrd}
\geometry{left=3.0cm,right=3.0cm,top=3.0cm,bottom=4.0cm}
\renewcommand{\thesection}{Problem \arabic{section}.}
\title{VP260 PROBLEM SET 4}
\author{Liu Yihao 515370910207}
\date{}
\begin{document}
\maketitle

\section{}
\begin{enumerate}[(a)]
\item
$$\frac{V_{AB}}{d}q=\rho\frac{4}{3}\pi r^3g$$
$$q=\frac{4\pi}{3}\frac{\rho r^3gd}{V_{AB}}$$
\item
$$6\pi\eta rv_\infty=\rho\frac{4}{3}\pi r^3g$$
$$r=3\sqrt{\frac{\eta v_\infty}{2\rho g}}$$
$$q=18\pi\frac{d}{V_{AB}}\sqrt{\frac{\eta^3v^3_\infty}{2\rho g}}$$
\item
$$r=3\sqrt{\frac{\eta v_\infty}{2\rho g}}=5.066\times10^{-7}m$$
$$q=18\pi\frac{d}{V_{AB}}\sqrt{\frac{\eta^3v^3_\infty}{2\rho g}}=4.80\times10^{-19}C$$
There is about 3 electrons
\end{enumerate}

\section{}
If we add a unit at the left, the equivalent capacitance won't change.
Suppose the equivalent capacitance to be $C'$
$$\frac{1}{\frac{1}{C'+C}+\frac{1}{C}+\frac{1}{C}}=\frac{1}{C'}$$
$$2C'^2+2C'C-C^2=0$$
$$C'=\frac{\sqrt{3}-1}{2}C$$

\section{}
\begin{enumerate}[(a)]
\item
$$U=\frac{Q^2}{2C}=\frac{Q^2x}{2\varepsilon_0A}$$
\item
$$U'=\frac{Q^2}{2C}=\frac{Q^2(x+dx)}{2\varepsilon_0A}$$
$$\Delta U=U'-U=\frac{Q^2dx}{2\varepsilon_0A}$$
\item
$$\frac{Q^2dx}{2\varepsilon_0A}=Fdx$$
$$F=\frac{Q^2}{2\varepsilon_0A}$$
\item
$$E=\frac{Q}{\varepsilon_0A}$$
$$F=\frac{1}{2}QE$$
Because the another $\frac{1}{2}QE$ is used to maintain the static plate.

\end{enumerate}

\section{}
\begin{enumerate}[(a)]
\item
$$C=\frac{\varepsilon_0L(L-x)}{D}+\frac{\varepsilon_r\varepsilon_0Lx}{D}=\frac{\varepsilon_0L}{D}(L-x+\varepsilon_rx)$$
\item
$$U=\frac{1}{2}CV^2=\frac{\varepsilon_0LV^2}{2D}(L-x+\varepsilon_rx)$$
$$U'=\frac{\varepsilon_0LV^2}{2D}[L-(x+dx)+\varepsilon_r(x+dx)]$$
$$dU=U'-U=\frac{(\varepsilon_r-1)\varepsilon_0V^2L}{2D}dx$$
\item
$$Q=CV=\frac{\varepsilon_0LV}{D}(L-x+\varepsilon_rx)$$
$$Q'=-\frac{\varepsilon_0LV}{D}(L-x+\varepsilon_rx)$$
$$U=\frac{Q^2}{2C}=\frac{Q^2D}{2\varepsilon_0L(L-x+\varepsilon_rx)}$$
$$dU=-\frac{Q^2D}{2\varepsilon_0L(L-x+\varepsilon_rx)^2}(\varepsilon_r-1)dx=-\frac{(\varepsilon_r-1)\varepsilon_0V^2L}{2D}dx$$
\item
$$F_b=-\frac{dU}{dx}=-\frac{(\varepsilon_r-1)\varepsilon_0V^2L}{2D}$$
So it pushes out the slab.
$$F_c=-\frac{dU}{dx}=\frac{(\varepsilon_r-1)\varepsilon_0V^2L}{2D}$$
So it pulls out the slab.
\item
The change of stored energy means that work should be done to the capacitor, the direction of the force is same to the change of stored energy, so it's wrong.
$$|F|=\frac{(\varepsilon_r-1)\varepsilon_0V^2L}{2D}$$
\end{enumerate}

\section{}
$$j(r)=\frac{I}{2\pi ra}$$
$$E(r)=\frac{j(r)}{\sigma}=\frac{I}{2\pi r\sigma a}$$
$$U=\int_{r_0}^{b-r_0}Edr=\int_{r_0}^{b-r_0}\frac{I}{2\pi r\sigma a}dr=\frac{I}{2\pi\sigma a}ln(\frac{b-r_0}{r_0})$$
$$R=2\frac{U}{I}=\frac{1}{\pi\sigma a}ln(\frac{b-r_0}{r_0})$$

\section{}
\begin{enumerate}[(a)]
\item
$$j(r)=\frac{I}{4\pi r^2}$$
$$E(r)=\frac{j(r)}{\sigma}=\frac{I}{4\pi r^2\sigma}$$
$$V=\int_{r_a}^{r_b}Edr=\int_{r_a}^{r_b}\frac{I}{4\pi r^2\sigma}dr=\frac{I}{4\pi\sigma}\left(\frac{1}{r_a}-\frac{1}{r_b}\right)$$
$$I=\frac{4\pi\sigma r_ar_bV}{r_b-r_a}$$
\item
$$R=\frac{V}{I}=\frac{r_b-r_a}{4\pi\sigma r_ar_b}$$
\item
When $r_b\gg r_a$,
$$R=\frac{1}{4\pi\sigma r_a}$$
So $r_b$ can be ignored when $r_b\gg r_a$
$$I=\frac{V}{2R}=8\pi\sigma r_aV$$
\end{enumerate}

\section{}
$$R'=\frac{1}{\frac{1}{4+6}+\frac{1}{8}+\frac{1}{9}}=\frac{360}{121}\Omega$$
$$R=\frac{1}{\frac{1}{20}+\frac{1}{3+360/121}}=4.6\Omega$$

\section{}
The voltages on the three points near point A are the same, so are the three points near point B.\\
So the three resistance between A and the three points near it is connected paralleled, and so are the three points near B and other six points.
$$R'=\frac{1}{3}R+\frac{1}{6}R+\frac{1}{3}R=\frac{5}{6}R$$

\section{}
Separate A and B, suppose the total current flowing from either point to be $I$, then the current flowing to four directions should be $\frac{I}{4}$.\\
For point A, a current of $\frac{I}{4}$ is flowing out for B.\\
For point B, a current of $\frac{I}{4}$ is flowing into B from A.\\
So the total current is $\frac{I}{2}$, and $R'=\frac{R}{2}$

\end{document}
