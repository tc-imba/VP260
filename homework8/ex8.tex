\documentclass{article}
\usepackage{enumerate}
\usepackage{amsmath}
\usepackage{amssymb}
\usepackage{graphicx}
\usepackage{subfigure}
\usepackage{geometry}
\usepackage{caption}
\usepackage{stmaryrd}
\geometry{left=3.0cm,right=3.0cm,top=3.0cm,bottom=4.0cm}
\renewcommand{\thesection}{Problem \arabic{section}.}
\title{VP260 PROBLEM SET 8}
\author{Liu Yihao 515370910207}
\date{}
\begin{document}
\maketitle

\section{}
$$B(r)=\frac{\mu_0 I}{2\pi r}$$
$$|U_{ab}|=\int_d^{d+L}\frac{\mu_0 I}{2\pi r}\cdot vdr=\frac{\mu_0 Iv}{2\pi}\ln\frac{d+L}{d}$$
According to right-hand rule, point A is at higher potential.

\section{}
\begin{enumerate}[(a)]
\item
$$E=BLv\cos\varphi$$
$$F=BIL=\frac{BEL}{R}=\frac{B^2L^2v\cos\varphi}{R}$$
$$F\cos\varphi=mg\sin\varphi$$
$$v=\frac{mgR\sin\varphi}{B^2L^2v\cos^2\varphi}$$
\item
According to right-hand rule, the direction is from A to B.
\item
$$BIL\cos\varphi=mg\sin\varphi$$
$$I=\frac{mg\tan\varphi}{BL}$$
\item
$$P=I^2R=\frac{m^2g^2R\tan^2\varphi}{B^2L^2}$$
\item
$$P'=mgv\sin\varphi=\frac{m^2g^2R\sin^2\varphi}{B^2L^2\cos^2\varphi}=\frac{m^2g^2R\tan^2\varphi}{B^2L^2}$$
The value of $P$ is the same.
\end{enumerate}

\section{}
$$\Phi=\int_0^a\int_0^a4t^2ydydx=2a^3t^2$$
$$\varepsilon=-\frac{d\Phi}{dt}=-4a^3t$$
So the direction of $I$ is clockwise.

\section{}
\begin{enumerate}[(a)]
\item
$$B(r)=\frac{\mu_0 I}{2\pi r}$$
$$\Phi=\int_0^a\int_{d+vt}^{b+d+vt}\frac{\mu_0 I}{2\pi r}drdl=\frac{\mu_0 Ia}{2\pi}\ln\left(1+\frac{b}{d+vt}\right)$$
\item
$$I=\frac{\varepsilon}{R}=-\frac{d\Phi}{Rdt}=-\frac{\mu_0 Ia}{2\pi R}\frac{d+vt}{b+d+vt}\frac{-bv}{(d+vt)^2}=\frac{\mu_0 Iabv}{2\pi R(d+vt)(b+d+vt)}$$
So the direction of $I$ is anti-clockwise.
\end{enumerate}

\section{}
The flux in the square loop decreased, so the force pointed to outside the loop, according to left-hand rule, the direction of $I$ is anti-clockwise.
$$B(r)=\frac{\mu_0 I}{2\pi r}$$
$$\Phi_0=\int_0^a\int_s^{s+a}\frac{\mu_0 I}{2\pi r}drdx=\frac{\mu_0 Ia}{2\pi}\ln\frac{s+a}{s}$$
$$I=\frac{\varepsilon}{R}=-\frac{d\Phi}{Rdt}$$
$$Q=I\int Idt=-\frac{0-\Phi_0}{R}=\frac{\Phi_0}{R}=\frac{\mu_0 Ia}{2\pi R}\ln\frac{s+a}{s}$$

\section{}
\begin{enumerate}[(a)]
\item
$$\frac{Q}{C}+R\cdot\frac{dQ}{dt}=0$$
$$C=\frac{\varepsilon_0\varepsilon_rA}{d},\quad R=\frac{\rho d}{A},\quad Q(0)=Q_0$$
$$\frac{dQ}{dt}+\frac{Q}{RC}=0$$
$$Q=Q_0e^{-t/RC}=Q_0e^{-t/\rho\varepsilon_0\varepsilon_r}$$
$$I=\frac{dQ}{dt}=-\frac{Q_0}{\rho\varepsilon_0\varepsilon_r}Q_0e^{-t/\rho\varepsilon_0\varepsilon_r}$$
$$J_c=\frac{I}{A}=-\frac{Q_0}{\rho A\varepsilon_0\varepsilon_r}e^{-t/\rho\varepsilon_0\varepsilon_r}$$
\item
$$J_d=\frac{E}{\rho}=\frac{Q_0d}{\rho A\varepsilon_0\varepsilon_r}e^{-t/\rho\varepsilon_0\varepsilon_r}$$
$$J_c+J_d=0$$
\end{enumerate}

\section{}
In the side AB, we can find $U_{AB}=E\cdot AB$\\
In other three sides, we can simply find $U=0$\\
So in the loop, we can find $\varepsilon=E\cdot AB$\\
Since the magnetic field is constant, $\frac{d\Phi}{dt}=0$\\
And $\varepsilon=-\frac{d\Phi}{dt}=0$, which reaches a contradiction.

\section{}
\begin{enumerate}[(a)]
\item
$$A(r)=\frac{1}{2}(B\times r)=\frac{1}{2}B\hat{n_z}\times(x\hat{n_x}+y\hat{n_y}+z\hat{n_z})=\frac{1}{2}(-By\hat{n_x}+Bx\hat{n_y})$$
\begin{align*}
B=\nabla\times A
&=\left(\frac{\partial A_z}{\partial y}-\frac{\partial A_y}{\partial z}\right)\hat{n_x}+\left(\frac{\partial A_x}{\partial z}-\frac{\partial A_z}{\partial x}\right)\hat{n_y}+\left(\frac{\partial A_y}{\partial x}-\frac{\partial A_x}{\partial y}\right)\hat{n_z}\\
&=\frac{1}{2}(B+B)\hat{n_z}\\
&=B\hat{n_z}
\end{align*}
$$div\ A=\frac{\partial A_x}{\partial x}+\frac{\partial A_y}{\partial y}+\frac{\partial A_z}{\partial z}=0+0=0$$
\item
\begin{align*}
B=\nabla\times A
&=\left(\frac{\partial A_z}{\partial y}-\frac{\partial A_y}{\partial z}\right)\hat{n_x}+\left(\frac{\partial A_x}{\partial z}-\frac{\partial A_z}{\partial x}\right)\hat{n_y}+\left(\frac{\partial A_y}{\partial x}-\frac{\partial A_x}{\partial y}\right)\hat{n_z}\\
&=(0-0)\hat{n_x}+(0-0)\hat{n_y}+(B-0)\hat{n_z}\\
&=B\hat{n_z}
\end{align*}
$$div\ A=\frac{\partial A_x}{\partial x}+\frac{\partial A_y}{\partial y}+\frac{\partial A_z}{\partial z}=0$$

\end{enumerate}

\section{}
\begin{enumerate}[(a)]
\item
$$\nabla^2A=\nabla(\nabla\cdot A)-\nabla\times(\nabla\times A)=0-\nabla\times B=-\mu_0J$$
\item
The vector potential is defined as
$$A(r)=\frac{1}{4\pi}\int\frac{\nabla\times v(r')}{|r-r'|}d\tau'$$
where $v(r')=B(r')$ which is a solenoidal vector field.\\
Since $\nabla\times B(r')=\mu_0J(r')$
$$A(r)=\frac{\mu_0}{4\pi}\int\frac{J(r')}{|r-r'|}d\tau'$$
\end{enumerate}

\end{document}
