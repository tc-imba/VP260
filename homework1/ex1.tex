\documentclass{article}
\usepackage{enumerate}
\usepackage{amsmath}
\usepackage{amssymb}
\usepackage{graphicx}
\usepackage{subfigure}
\usepackage{geometry}
\usepackage{caption}
\geometry{left=3.0cm,right=3.0cm,top=3.0cm,bottom=4.0cm}
\renewcommand{\thesection}{Problem \arabic{section}.}
\title{VP260 PROBLEM SET 1}
\author{Liu Yihao 515370910207}
\date{}
\begin{document}
\maketitle

\section{}
$$E=\frac{1}{4\pi\varepsilon_0}\int_{\theta_1}^{\theta_2}\frac{Q(\cos\theta\hat{n_x}+\sin\theta\hat{n_y})}{r^2(\theta_2-\theta_1)}d\theta
=\frac{Q[(\sin\theta_2-\sin\theta_1)\hat{n_x}+(\cos\theta_1-\cos\theta_2)\hat{n_y}]}{4\pi\varepsilon_0r^2(\theta_2-\theta_1)}$$
$$|E|=\frac{-Q}{4\pi\varepsilon_0r^2(\theta_2-\theta_1)}\sqrt{2-2\sin\theta_1\sin\theta_2-2\cos\theta_1\cos\theta_2}=\frac{-Q}{2\pi\varepsilon_0r^2}\cdot\frac{\sin\frac{\theta_1+\theta_2}{2}}{\theta_2-\theta_1}$$
\begin{align*}
&|E_A|=\frac{-Q}{2\pi\varepsilon_0r^2}\cdot\frac{2\sqrt{2}}{3\pi}
&|E_B|=\frac{-Q}{2\pi\varepsilon_0r^2}\cdot\frac{2}{\pi}\\
&|E_C|=\frac{-Q}{2\pi\varepsilon_0r^2}\cdot\frac{2\sqrt{2}}{\pi}
&|E_D|=0
\end{align*}
$$|E_D|<|E_A|<|E_B|<|E_C|$$

\section{}
Suppose right to be the direction of x-axis and up to be the direction of the y-axis.\\
\\
At point A,
$$E_1=-\frac{1}{4\pi\varepsilon_0}\cdot\frac{4q}{a^2}\,\hat{n_y}\quad
E_2=\frac{1}{4\pi\varepsilon_0}\cdot\frac{8q}{a^2}\,\hat{n_y}$$
$$E_3=\frac{1}{4\pi\varepsilon_0}\cdot\frac{4q}{5a^2}\,\left(\frac{2\sqrt{5}}{5}\hat{n_x}-\frac{\sqrt{5}}{5}\hat{n_y}\right)$$
$$E_A=\frac{q}{5\pi\varepsilon_0a^2}\left(\frac{2\sqrt{5}}{5}\hat{n_x}+\frac{25-\sqrt{5}}{5}\hat{n_y}\right)
=(6.43\times10^{20}\,\hat{n_x}+3.27\times10^{21}\,\hat{n_y})\,\rm{N/C}$$
At point B,
$$E_1=\frac{1}{4\pi\varepsilon_0}\cdot\frac{q}{a^2}\,\hat{n_x}\quad
E_2=-\frac{1}{4\pi\varepsilon_0}\cdot\frac{q}{a^2}\,\hat{n_y}$$
$$E_3=\frac{1}{4\pi\varepsilon_0}\cdot\frac{q}{a^2}\,\left(\frac{\sqrt{2}}{2}\hat{n_x}+\frac{\sqrt{2}}{2}\hat{n_y}\right)$$
$$E_B=\frac{q}{4\pi\varepsilon_0a^2}\left(\frac{\sqrt{2}+2}{2}\hat{n_x}+\frac{\sqrt{2}-2}{2}\hat{n_y}\right)
=(1.53\times10^{21}\,\hat{n_x}-2.63\times10^{21}\,\hat{n_y})\,\rm{N/C}$$
At point C,
$$E_1=E_2=\frac{1}{4\pi\varepsilon_0}\cdot\frac{2q}{a^2}\,\left(\frac{\sqrt{2}}{2}\hat{n_x}-\frac{\sqrt{2}}{2}\hat{n_y}\right)$$
$$E_3=\frac{1}{4\pi\varepsilon_0}\cdot\frac{4q}{a^2}\,\left(\frac{\sqrt{2}}{2}\hat{n_x}+\frac{\sqrt{2}}{2}\hat{n_y}\right)$$
$$E_C=\frac{\sqrt{2}q}{\pi\varepsilon_0a^2}\,\hat{n_x}
=5.08\times10^{21}\,\hat{n_x}\,\rm{N/C}$$
\section{}
Suppose top to be the positive direction,\\
\\
The two charges on the left will give a electric field
$$E_1=-\frac{1}{4\pi\varepsilon_0}\cdot\frac{2q}{(r+a/2)^2+a^2/4}\cdot\frac{a/2}{\sqrt{(r+a/2)^2+a^2/4}}$$
The two charges on the right will give a electric field
$$E_2=\frac{1}{4\pi\varepsilon_0}\cdot\frac{2q}{(r-a/2)^2+a^2/4}\cdot\frac{a/2}{\sqrt{(r-a/2)^2+a^2/4}}$$
As $r\gg a$,
\begin{align*}
E=E_1+E_2&\approx\frac{qa}{4\pi\varepsilon_0}\left(\frac{1}{(r-a/2)^3}-\frac{1}{(r+a/2)^3}\right)\\
&=\frac{qa}{4\pi\varepsilon_0}\cdot\frac{3r^2a+a^3/4}{(r^2-a^2/4)^3}
\approx\frac{3qa^2}{4\pi\varepsilon_0r^4}
\end{align*}

\section{}
\begin{enumerate}[(a)]
\item
$$E=\int_0^l\frac{1}{4\pi\varepsilon_0}\cdot\frac{\lambda}{(l+a-x)^2}dx=\frac{\lambda}{4\pi\varepsilon_0}\cdot\frac{1}{l+a-x}\bigg|_0^l=\frac{\lambda}{4\pi\varepsilon_0}\left(\frac{1}{a}-\frac{1}{l+a}\right)
=\frac{\lambda l}{4\pi\varepsilon_0a(l+a)}$$
\item
$$E=\int_0^l\frac{1}{4\pi\varepsilon_0}\cdot\frac{Ax}{(l+a-x)^2}dx
=\frac{A}{4\pi\varepsilon_0}\left(\frac{l+a}{l+a-x}+ln(l+a-x)\right)\bigg|_0^l
=\frac{A}{4\pi\varepsilon_0}\left(\frac{l}{a}+ln\frac{a}{l+a}\right)$$
\end{enumerate}

\section{}
\begin{enumerate}[(a)]
\item
$$E=\int_{R_1}^{R_2}\frac{1}{4\pi\varepsilon_0}\cdot\frac{x\sigma2\pi rdr}{(x^2+r^2)^{3/2}}
=\frac{\sigma x}{2\varepsilon_0}\int_{x^2+R_1^2}^{x^2+R_2^2}\frac{du}{2u^{3/2}}
=\frac{\sigma x}{2\varepsilon_0}\left(\frac{1}{\sqrt{x^2+R_1^2}}-\frac{1}{\sqrt{x^2+R_2^2}}\right)$$

\item
When $x\ll R_1<R_2$,
$$E\approx\frac{\sigma x}{2\varepsilon_0}\left(\frac{1}{R_1}-\frac{1}{R_2}\right)$$
It is approximately proportional to the distance from the center.
$$mg=\frac{q\sigma x_0}{2\varepsilon_0}\left(\frac{1}{R_1}-\frac{1}{R_2}\right)$$
$$F=\frac{q\sigma(x-x_0)}{2\varepsilon_0}\left(\frac{1}{R_1}-\frac{1}{R_2}\right)$$
$$k=\frac{q\sigma}{2\varepsilon_0}\left(\frac{1}{R_1}-\frac{1}{R_2}\right)$$
$$T=2\pi\sqrt{\frac{m}{k}}=2\pi\sqrt{\frac{2m\varepsilon_0R_1R_2}{q\sigma(R_2-R_1)}}$$
\end{enumerate}

\section{}
\begin{enumerate}[(a)]
\item
\begin{align*}
F&=\frac{1}{4\pi\varepsilon_0}\int_0^l\int_0^l\frac{Q\frac{dx}{l}\cdot Q\frac{dy}{l}}{(a+x+y)^2}\\
&=\frac{Q^2}{4\pi\varepsilon_0l^2}\int_0^l\int_0^l\frac{1}{(a+x+y)^2}dydx\\
&=\frac{Q^2}{4\pi\varepsilon_0l^2}\int_0^l\left(\frac{1}{a+x}-\frac{1}{a+x+l}\right)dx\\
&=\frac{Q^2}{4\pi\varepsilon_0l^2}ln\frac{a+x}{a+x+l}\bigg|_0^l\\
&=\frac{Q^2}{4\pi\varepsilon_0l^2}ln\left[\frac{(a+l)^2}{a(a+2l)}\right]
\end{align*}

\item
$$\lim_{x\to 0}\frac{ln(x+1)}{x}=1$$
$$ln(x+1)\approx x\ when\ x\to 0$$
$$\frac{(a+l)^2}{a(a+2l)}=1+\frac{l^2}{a^2+2al}$$
If $a\gg l$,
$$ln\left[\frac{(a+l)^2}{a(a+2l)}\right]\approx\frac{l^2}{a^2+2al}$$
$$F\approx\frac{Q^2}{4\pi\varepsilon_0l^2}\cdot\frac{l^2}{a^2+2al}
=\frac{Q^2}{4\pi\varepsilon_0(a^2+2al)}
\approx\frac{Q^2}{4\pi\varepsilon_0a^2}$$
\end{enumerate}

\end{document}
